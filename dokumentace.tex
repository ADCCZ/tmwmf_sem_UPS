\documentclass[12pt,a4paper]{article}
\usepackage[utf8]{inputenc}
\usepackage[czech]{babel}
\usepackage[T1]{fontenc}
\usepackage{geometry}
\geometry{margin=2.5cm}
\usepackage{graphicx}
\usepackage{listings}
\usepackage{xcolor}
\usepackage{hyperref}
\usepackage{enumitem}
\usepackage{float}
\usepackage{fancyhdr}
\usepackage{amssymb}  % Pro \checkmark
\usepackage{pifont}    % Pro \ding{51} (alternative checkmark)
\usepackage{tikz}      % Pro stavové diagramy
\usetikzlibrary{arrows.meta,positioning,shapes.geometric,automata}

% Definice vlastního checkmark (fallback)
\newcommand{\OK}{$\checkmark$}

% Nastavení pro listings
\lstset{
    basicstyle=\ttfamily\small,
    breaklines=true,
    frame=single,
    numbers=left,
    numberstyle=\tiny,
    tabsize=4,
    showstringspaces=false,
    commentstyle=\color{gray},
    keywordstyle=\color{blue},
    stringstyle=\color{red}
}

% Header a footer
\setlength{\headheight}{15pt}  % Fix fancyhdr warning
\pagestyle{fancy}
\fancyhf{}
\rhead{KIV/UPS - Pexeso}
\lhead{Semestrální práce}
\rfoot{Strana \thepage}

\hypersetup{
    colorlinks=true,
    linkcolor=blue,
    filecolor=magenta,
    urlcolor=cyan,
    pdftitle={Dokumentace - Pexeso},
}

\begin{document}

% Titulní strana
\begin{titlepage}
    \centering
    \vspace*{2cm}

    {\Huge\bfseries Síťová hra Pexeso\par}
    \vspace{1cm}
    {\Large Semestrální práce z předmětu KIV/UPS\par}
    \vspace{2cm}

    {\Large Architektura server-klient (1:N)\par}
    \vspace{0.5cm}
    {\large Textový protokol přes TCP\par}
    \vspace{3cm}

    {\large
    Server: C (BSD sockets, POSIX threads)\\
    Klient: Java (JavaFX, java.net.Socket)\\
    }

    \vfill

    {\large Západočeská univerzita v Plzni\\
    Fakulta aplikovaných věd\\
    \today\par}
\end{titlepage}

\tableofcontents
\newpage

% 1. ÚVOD
\section{Úvod}

Tato dokumentace popisuje implementaci síťové hry Memory (Pexeso) jako semestrální práce z předmětu KIV/UPS. Projekt implementuje kompletní client-server architekturu s podporou multiplayer režimu (2--4 hráči), lobby systémem, automatickým reconnect mechanismem a robustním textovým protokolem přes TCP.

\subsection{Popis hry}

Pexeso je tahová paměťová hra pro 2--4 hráče. Herní deska obsahuje sudý počet karet (typicky 16, 24, 32 nebo 36 karet) rozmístěných lícem dolů. Každá karta má přidělenou hodnotu (symbol) a každá hodnota se vyskytuje přesně 2×.

\textbf{Pravidla:}
\begin{itemize}[noitemsep]
    \item Hráči se střídají v tazích
    \item V každém tahu hráč otočí postupně 2 karty
    \item Pokud se hodnoty karet shodují $\rightarrow$ hráč získává 1 bod a hraje znovu
    \item Pokud se neshodují $\rightarrow$ karty se vrátí lícem dolů a přichází další hráč
    \item Hra končí, když jsou všechny páry nalezeny
    \item Vyhrává hráč(i) s nejvyšším skóre
\end{itemize}

\subsection{Architektura systému}

Systém se skládá ze dvou hlavních komponent:

\begin{description}[leftmargin=3cm]
    \item[Server (C)] Spravuje herní logiku, místnosti, připojené klienty a stav her. Implementován v jazyce C s využitím BSD sockets a POSIX threads.
    \item[Klient (Java)] Grafická aplikace s JavaFX pro zobrazení hry a interakci s uživatelem. Komunikuje se serverem pomocí textového protokolu.
\end{description}

\textbf{Klíčové vlastnosti:}
\begin{itemize}[noitemsep]
    \item Textový protokol (ASCII) přes TCP
    \item Podpora 2--4 hráčů současně
    \item Lobby systém s výběrem místností
    \item Automatický reconnect po výpadku
    \item PING/PONG keepalive mechanismus
    \item Validace všech síťových zpráv
    \item Logování na serveru i klientovi
    \item Paralelní běh více herních místností
\end{itemize}

\newpage

% 2. KOMUNIKAČNÍ PROTOKOL
\section{Komunikační protokol}

\subsection{Základní charakteristika}

Protokol je navržen jako \textbf{textový aplikační protokol} nad transportním protokolem TCP. Všechny zprávy jsou kódovány v ASCII bez diakritiky.

\begin{table}[H]
\centering
\begin{tabular}{|l|l|}
\hline
\textbf{Parametr} & \textbf{Hodnota} \\
\hline
Typ protokolu & Textový, aplikační vrstva \\
Transportní protokol & TCP \\
Kódování & ASCII (bez diakritiky) \\
Ukončení zprávy & \texttt{\textbackslash n} (newline) \\
Formát zprávy & \texttt{COMMAND [PARAM1] [PARAM2] ...} \\
Oddělování parametrů & Jedna mezera (ASCII 0x20) \\
\hline
\end{tabular}
\caption{Základní parametry protokolu}
\end{table}

\subsection{Datové typy a omezení}

\begin{table}[H]
\centering
\small
\begin{tabular}{|l|p{4cm}|p{3cm}|p{3cm}|}
\hline
\textbf{Typ} & \textbf{Popis} & \textbf{Formát} & \textbf{Omezení} \\
\hline
nickname & Přezdívka hráče & String bez mezer & 1--16 znaků -> \newline a-z, A-Z, 0-9,  - \\
room\_id & ID místnosti & Integer & 1--9999 \\
room\_name & Název místnosti & String bez mezer & 1--20 znaků \\
client\_id & ID klienta & Integer & 1--9999 \\
card\_id & Index karty & Integer & 0--(board\_size-1) \\
card\_value & Hodnota karty & Integer & 0--(board\_size/2-1) \\
board\_size & Počet karet & Integer & 16, 24, 32, 36 \\
max\_players & Max. hráčů & Integer & 2--4 \\
score & Skóre hráče & Integer & $\geq$ 0 \\
\hline
\end{tabular}
\caption{Datové typy protokolu}
\end{table}

\subsection{Příklady zpráv}

\textbf{Klient $\rightarrow$ Server:}
\begin{lstlisting}[language=bash]
HELLO Alice
LIST_ROOMS
CREATE_ROOM Game1 2 16
JOIN_ROOM 42
START_GAME
FLIP 5
PONG
RECONNECT 123
\end{lstlisting}

\newpage

\textbf{Server $\rightarrow$ Klient:}
\begin{lstlisting}[language=bash]
WELCOME 1
ROOM_LIST 2 1 Game1 2 4 WAITING 2 Game2 1 2 PLAYING
ROOM_CREATED 1 Game1
GAME_START 16 2 Alice Bob
YOUR_TURN
CARD_REVEAL 5 3 Alice
MATCH Alice 1
MISMATCH Bob
GAME_END Alice 8 Bob 6
PING
PLAYER_DISCONNECTED Bob SHORT
ERROR INVALID_COMMAND Unknown command
\end{lstlisting}

\subsection{Stavový diagram klienta}

Klient prochází následujícími stavy:

\begin{enumerate}[noitemsep]
    \item \texttt{DISCONNECTED} -- Nepřipojeno k serveru
    \item \texttt{CONNECTED} -- TCP spojení navázáno
    \item \texttt{AUTHENTICATED} -- Po úspěšném \texttt{HELLO}
    \item \texttt{IN\_LOBBY} -- V lobby (může listovat místnosti)
    \item \texttt{IN\_ROOM} -- V místnosti (čeká na start)
    \item \texttt{IN\_GAME} -- Hra běží
    \item \texttt{SHORT\_DISCONNECT} -- Krátkodobý výpadek (< 90s)
    \item \texttt{LONG\_DISCONNECT} -- Dlouhodobý výpadek (> 90s)
\end{enumerate}

Přechody mezi stavy jsou řízeny protokolovými zprávami (např. \texttt{WELCOME}, \texttt{ROOM\_JOINED}, \texttt{GAME\_START}).

\begin{figure}[H]
\centering
\begin{tikzpicture}[
    node distance=3.5cm,
    auto,
    >=Stealth,
    state/.style={
        rectangle,
        rounded corners,
        draw=black,
        thick,
        minimum width=3.2cm,
        minimum height=0.9cm,
        text centered,
        font=\footnotesize
    },
    initial/.style={
        state,
        fill=red!20
    },
    final/.style={
        state,
        fill=green!20
    }
]

% HORNÍ ŘADA - Připojení a lobby
\node[initial] (disc) at (0,0) {DISCONNECTED};
\node[state] (conn) at (4,0) {CONNECTED};
\node[state] (auth) at (8,0) {AUTHENTICATED};
\node[state] (lobby) at (12,0) {IN\_LOBBY};

% STŘEDNÍ ŘADA - Room a hra
\node[state] (room) at (8,-3) {IN\_ROOM};
\node[state] (game) at (4,-3) {IN\_GAME};

% SPODNÍ ŘADA - Disconnect stavy
\node[state] (short) at (2,-6) {SHORT\_DISC};
\node[final] (long) at (10,-6) {LONG\_DISC};

% HLAVNÍ FLOW - Připojení
\draw[->, very thick] (disc) -- node[above] {\scriptsize connect()} (conn);
\draw[->, very thick] (conn) -- node[above] {\scriptsize HELLO} (auth);
\draw[->, very thick] (auth) -- node[above] {\scriptsize WELCOME} (lobby);

% HLAVNÍ FLOW - Hra
\draw[->, very thick] (lobby) -- node[right, pos=0.3] {\scriptsize JOIN\_ROOM} (room);
\draw[->, very thick] (room) -- node[above] {\scriptsize GAME\_START} (game);

% RECONNECT mechanismus
\draw[->, thick, dashed] (game) -- node[left, align=center] {\scriptsize timeout\\[-1mm]\scriptsize výpadek} (short);
\draw[->, thick, dashed] (short) to[bend left=20] node[below left, align=center] {\scriptsize RECONNECT\\[-1mm]\scriptsize +GAME\_STATE} (game);
\draw[->, thick] (short) -- node[above] {\scriptsize >90s} (long);

% NÁVRAT do lobby
\draw[->] (game) to[bend left=25] node[above, pos=0.6] {\scriptsize GAME\_END} (lobby);
\draw[->] (room) to[bend right=15] node[right, pos=0.7] {\scriptsize LEAVE\_ROOM} (lobby);

% ODPOJENÍ
\draw[->] (lobby) to[bend left=40] node[above, pos=0.5] {\scriptsize QUIT} (disc);
\draw[->] (game) to[bend right=30] node[below left, pos=0.3] {\scriptsize disconnect()} (disc);

% LIST ROOMS (loop)
\draw[->] (lobby) edge[loop right, looseness=8] node[right] {\scriptsize LIST\_ROOMS} (lobby);

\end{tikzpicture}
\caption{Stavový diagram klienta}
\label{fig:state-diagram-client}
\end{figure}

\subsection{Keepalive mechanismus (PING/PONG)}

Server pravidelně kontroluje živost klientů pomocí PING/PONG mechanismu:

\begin{itemize}[noitemsep]
    \item Server posílá \texttt{PING} každých 5 sekund po přijetí \texttt{PONG}
    \item Klient musí odpovědět \texttt{PONG} do 5 sekund
    \item Pokud klient neodpoví $\rightarrow$ server označí klienta jako odpojeného
    \item Klient má \texttt{READ\_TIMEOUT} 15 sekund -- pokud od serveru nepřijde žádná zpráva, detekuje výpadek
\end{itemize}

\subsection{Reconnect mechanismus}

\textbf{Krátkodobý výpadek (SHORT\_DISCONNECT):}
\begin{itemize}[noitemsep]
    \item Klient detekuje odpojení (READ\_TIMEOUT 15s)
    \item Automatický reconnect: 7 pokusů × 10s interval = 70s celkem
    \item Server čeká 90 sekund na reconnect
    \item Po úspěšném reconnectu server pošle \texttt{GAME\_STATE} s aktuálním stavem hry
    \item Hra pokračuje ze stejného stavu
\end{itemize}

\textbf{Dlouhodobý výpadek (LONG\_DISCONNECT):}
\begin{itemize}[noitemsep]
    \item Klient zkouší reconnect 70s, poté vzdá
    \item Server po 90s bez reconnectu odstraní hráče
    \item Server pošle \texttt{PLAYER\_DISCONNECTED <nick> LONG}
    \item Pokud zbývá < 2 hráči $\rightarrow$ hra končí (\texttt{GAME\_END\_FORFEIT})
    \item Pokud zbývá $\geq$ 2 hráči $\rightarrow$ hra pokračuje bez odpojeného hráče
\end{itemize}

\subsection{Návaznost zpráv (sekvenční diagram)}

Následující diagram ukazuje typickou komunikaci při připojení, vytvoření hry a hraní:

\begin{figure}[H]
\centering
\begin{tikzpicture}[
    >=Stealth,
    node distance=0.5cm,
    message/.style={->, thick},
    actor/.style={rectangle, draw, minimum width=2cm, minimum height=0.8cm}
]

% Aktéři
\node[actor] (alice) at (0,0) {Alice};
\node[actor] (server) at (5,0) {Server};
\node[actor] (bob) at (10,0) {Bob};

% Vertikální linie
\draw[thick] (alice.south) -- ++(0,-14);
\draw[thick] (server.south) -- ++(0,-14);
\draw[thick] (bob.south) -- ++(0,-14);

% Zprávy - Alice
\draw[message] (0,-1) -- node[above] {\tiny HELLO Alice} (5,-1);
\draw[message] (5,-1.5) -- node[above] {\tiny WELCOME 1} (0,-1.5);

\draw[message] (0,-2.5) -- node[above] {\tiny CREATE\_ROOM Game1 2 16} (5,-2.5);
\draw[message] (5,-3) -- node[above] {\tiny ROOM\_CREATED 1 Game1} (0,-3);

% Zprávy - Bob
\draw[message] (10,-4) -- node[above] {\tiny HELLO Bob} (5,-4);
\draw[message] (5,-4.5) -- node[above] {\tiny WELCOME 2} (10,-4.5);

\draw[message] (10,-5.5) -- node[above] {\tiny JOIN\_ROOM 1} (5,-5.5);
\draw[message] (5,-6) -- node[above] {\tiny ROOM\_JOINED 1 ...} (10,-6);
\draw[message] (5,-6.5) -- node[above] {\tiny PLAYER\_JOINED Bob} (0,-6.5);

% Start hry
\draw[message] (0,-7.5) -- node[above] {\tiny START\_GAME} (5,-7.5);
\draw[message] (5,-8) -- node[above] {\tiny GAME\_START 16 2 Alice Bob} (0,-8);
\draw[message] (5,-8.5) -- node[above] {\tiny GAME\_START 16 2 Alice Bob} (10,-8.5);

\draw[message] (5,-9) -- node[above] {\tiny YOUR\_TURN} (0,-9);

% Herní tah
\draw[message] (0,-10) -- node[above] {\tiny FLIP 5} (5,-10);
\draw[message] (5,-10.5) -- node[above] {\tiny CARD\_REVEAL 5 3 Alice} (0,-10.5);
\draw[message] (5,-11) -- node[above] {\tiny CARD\_REVEAL 5 3 Alice} (10,-11);

\draw[message] (0,-11.5) -- node[above] {\tiny FLIP 12} (5,-11.5);
\draw[message] (5,-12) -- node[above] {\tiny CARD\_REVEAL 12 7 Alice} (0,-12);
\draw[message] (5,-12.5) -- node[above] {\tiny CARD\_REVEAL 12 7 Alice} (10,-12.5);

\draw[message] (5,-13) -- node[above] {\tiny MISMATCH Bob} (0,-13);
\draw[message] (5,-13.5) -- node[above] {\tiny MISMATCH Bob} (10,-13.5);

\end{tikzpicture}
\caption{Sekvenční diagram typické komunikace (připojení, vytvoření hry, herní tah)}
\label{fig:sequence-diagram}
\end{figure}

\subsection{Validace a chybové stavy}

Server validuje všechny příchozí zprávy a odpovídá chybovými kódy:

\begin{table}[H]
\centering
\small
\begin{tabular}{|l|p{8cm}|}
\hline
\textbf{Error kód} & \textbf{Popis} \\
\hline
INVALID\_COMMAND & Neznámý příkaz \\
INVALID\_PARAMS & Špatný počet/formát parametrů \\
NOT\_AUTHENTICATED & Příkaz vyžaduje autentizaci \\
NICK\_IN\_USE & Přezdívka je již používána \\
ROOM\_NOT\_FOUND & Místnost neexistuje \\
ROOM\_FULL & Místnost je plná \\
NOT\_IN\_ROOM & Hráč není v místnosti \\
NOT\_ROOM\_OWNER & Pouze vlastník může startovat hru \\
NOT\_YOUR\_TURN & Není na tahu tento hráč \\
INVALID\_CARD & Neplatné ID karty \\
\hline
\end{tabular}
\caption{Chybové kódy protokolu}
\end{table}

Server odpojí klienta po \textbf{3 nevalidních zprávách} (konfigurovatelné).

\newpage

% 3. IMPLEMENTACE SERVERU
\section{Implementace serveru (C)}

\subsection{Architektura a moduly}

Server je implementován v jazyce C s následující modulární strukturou:

\begin{table}[H]
\centering
\small
\begin{tabular}{|l|p{9cm}|}
\hline
\textbf{Modul} & \textbf{Odpovědnost} \\
\hline
\texttt{main.c} & Vstupní bod, parsování argumentů, inicializace serveru \\
\texttt{server.c} & Listening socket, accept loop, vytváření threadů \\
\texttt{client\_handler.c} & Obsluha jednotlivých klientů, zpracování zpráv \\
\texttt{client\_list.c} & Správa seznamu připojených klientů \\
\texttt{room.c} & Správa lobby a herních místností \\
\texttt{game.c} & Logika hry Pexeso (deska, tahy, vyhodnocení) \\
\texttt{logger.c} & Thread-safe logování do souboru \\
\texttt{protocol.h} & Definice protokolových konstant \\
\hline
\end{tabular}
\caption{Moduly serveru}
\end{table}

\subsection{Datové struktury}

\textbf{Struktura klienta:}
\begin{lstlisting}[language=C]
typedef struct client_t {
    int client_id;           // Unikatni ID
    int socket_fd;           // Socket file descriptor
    char nickname[MAX_NICK_LENGTH];
    client_state_t state;    // CONNECTED/AUTHENTICATED/...
    time_t last_activity;    // Posledni aktivita
    int invalid_message_count; // Pocet chyb
    int is_disconnected;     // Flag odpojeni
    pthread_mutex_t mutex;   // Thread safety
} client_t;
\end{lstlisting}

\textbf{Struktura místnosti:}
\begin{lstlisting}[language=C]
typedef struct room_t {
    int room_id;
    char name[MAX_ROOM_NAME_LENGTH];
    client_t *players[MAX_PLAYERS_PER_ROOM];
    int player_count;
    int max_players;
    client_t *owner;
    room_state_t state;      // WAITING/PLAYING
    game_t *game;            // Hra (pokud bezi)
    pthread_mutex_t mutex;
} room_t;
\end{lstlisting}

\newpage

\textbf{Struktura hry:}
\begin{lstlisting}[language=C]
typedef struct game_t {
    int board_size;          // Pocet karet
    card_t *cards;           // Pole karet
    client_t **players;      // Hraci
    int player_count;
    int current_player;      // Index aktualniho hrace
    int *scores;             // Skore hracu
    int flipped_count;       // Pocet otevenych karet
    int first_card_id;       // ID prvni karty
} game_t;
\end{lstlisting}

\subsection{Paralelizace -- POSIX threads}

Server používá \textbf{thread-per-client} model:

\begin{itemize}[noitemsep]
    \item Hlavní thread běží v \texttt{server\_run()} -- nekonečná accept loop
    \item Pro každého nového klienta se vytvoří vlastní thread (\texttt{pthread\_create})
    \item Thread se detachuje (\texttt{pthread\_detach}) $\rightarrow$ automatický cleanup
    \item Každý thread má vlastní \texttt{client\_t} strukturu
    \item Logger používá \texttt{pthread\_mutex\_t} pro thread-safe zápis
    \item Room broadcast operace jsou synchronizované mutexem
\end{itemize}

\textbf{Výhody thread-per-client:}
\begin{itemize}[noitemsep]
    \item Jednoduchá implementace -- každý klient má vlastní kontext
    \item Blokující I/O operace neovlivňují ostatní klienty
    \item Přirozená izolace mezi klienty
    \item Snadný debugging
\end{itemize}

\subsection{Použité knihovny a API}

\textbf{BSD Sockets (POSIX):}
\begin{lstlisting}[language=C]
#include <sys/socket.h>   // socket(), bind(), listen(), accept()
#include <netinet/in.h>   // struct sockaddr_in
#include <arpa/inet.h>    // inet_pton(), inet_ntop()
\end{lstlisting}

\textbf{POSIX Threads:}
\begin{lstlisting}[language=C]
#include <pthread.h>      // pthread_create(), pthread_mutex_t
\end{lstlisting}

\textbf{Standardní knihovna:}
\begin{lstlisting}[language=C]
#include <unistd.h>       // close(), read(), write()
#include <string.h>       // strcpy(), strncpy(), strcmp()
#include <stdlib.h>       // malloc(), free(), atoi()
#include <stdio.h>        // printf(), snprintf(), fopen()
#include <time.h>         // time(), localtime()
\end{lstlisting}

\textbf{Všechny použité knihovny jsou součástí POSIX standardu.} Nebyly použity žádné externí networking knihovny.

\subsection{Konfigurace serveru}

Server přijímá konfiguraci z příkazové řádky:

\begin{lstlisting}[language=bash]
./server <IP> <PORT> <MAX_ROOMS> <MAX_CLIENTS>
\end{lstlisting}

\textbf{Příklad:}
\begin{lstlisting}[language=bash]
./server 0.0.0.0 10000 10 50
\end{lstlisting}

Parametry:
\begin{itemize}[noitemsep]
    \item \texttt{IP} -- IP adresa pro bind (např. \texttt{0.0.0.0} pro všechna rozhraní)
    \item \texttt{PORT} -- Port pro listening (např. \texttt{10000})
    \item \texttt{MAX\_ROOMS} -- Maximální počet místností (např. \texttt{10})
    \item \texttt{MAX\_CLIENTS} -- Maximální počet klientů celkem (např. \texttt{50})
\end{itemize}

\subsection{Logování}

Server loguje všechny důležité události do souboru \texttt{server.log}:

\begin{itemize}[noitemsep]
    \item Připojení/odpojení klientů (IP, port, čas)
    \item Autentizace (nickname, client\_id)
    \item Vytvoření a zrušení místností
    \item Start a konec her
    \item Výpadky (SHORT/LONG disconnect)
    \item Reconnect události
    \item Nevalidní zprávy (příkaz, client\_id)
    \item Interní chyby
\end{itemize}

\textbf{Příklad logu:}
\begin{lstlisting}
[2026-01-11 12:00:00] [INFO] Server initialized on 0.0.0.0:10000
[2026-01-11 12:00:05] [INFO] Client 1 connected from 127.0.0.1:54321
[2026-01-11 12:00:06] [INFO] Client 1 authenticated as 'Alice'
[2026-01-11 12:00:10] [INFO] Room 1 created: Game1 (max 2 players)
[2026-01-11 12:00:15] [INFO] Client 2 joined room 1
[2026-01-11 12:00:20] [INFO] Game started in room 1
[2026-01-11 12:01:30] [WARNING] Client 1: Invalid command 'INVALID'
[2026-01-11 12:02:00] [INFO] Game ended in room 1: Alice 8, Bob 6
\end{lstlisting}

\newpage

% 4. IMPLEMENTACE KLIENTA
\section{Implementace klienta (Java + JavaFX)}

\subsection{Architektura MVC}

Klient je implementován v Javě s využitím JavaFX a architekturou Model-View-Controller:

\begin{table}[H]
\centering
\small
\begin{tabular}{|l|p{9cm}|}
\hline
\textbf{Balíček/Třída} & \textbf{Odpovědnost} \\
\hline
\texttt{Main.java} & Vstupní bod, správa scén, inicializace loggeru \\
\texttt{network/ClientConnection} & TCP spojení, async čtení, auto-reconnect \\
\texttt{network/MessageListener} & Rozhraní pro příjem zpráv \\
\texttt{protocol/ProtocolConstants} & Definice protokolu, timeouty \\
\texttt{model/Room} & Model herní místnosti \\
\texttt{controller/LoginController} & Ovládání přihlašovací obrazovky \\
\texttt{controller/LobbyController} & Ovládání lobby (seznam místností) \\
\texttt{controller/GameController} & Ovládání herní obrazovky \\
\texttt{util/Logger} & Thread-safe logování do \texttt{client.log} \\
\hline
\end{tabular}
\caption{Moduly klienta}
\end{table}

\subsection{Síťová komunikace}

\textbf{Asynchronní model:}

Klient používá oddělené vlákno pro síťovou komunikaci, aby GUI nezamrzalo:

\begin{lstlisting}[language=Java]
// Cteni v oddelnem vlakne
private void readerLoop() {
    while (running && (line = in.readLine()) != null) {
        final String message = line.trim();
        if (listener != null) {
            // GUI aktualizace pres Platform.runLater()
            listener.onMessageReceived(message);
        }
    }
}
\end{lstlisting}

\textbf{Schéma komunikace:}
\begin{verbatim}
[GUI Thread] -> sendMessage() -> socket
[Reader Thread] -> readLine() -> Platform.runLater() -> onMessageReceived()
\end{verbatim}

\subsection{Použité knihovny a API}

\textbf{Java Standard Library (java.net):}
\begin{lstlisting}[language=Java]
import java.net.Socket;              // TCP socket
import java.net.InetSocketAddress;   // Adresa serveru
\end{lstlisting}

\textbf{Java I/O:}
\begin{lstlisting}[language=Java]
import java.io.BufferedReader;       // Cteni textu
import java.io.InputStreamReader;    // Stream -> Reader
import java.io.PrintWriter;          // Zapis textu
\end{lstlisting}

\newpage

\textbf{JavaFX (GUI):}
\begin{lstlisting}[language=Java]
import javafx.application.Application;
import javafx.application.Platform;  // runLater()
import javafx.scene.Scene;
import javafx.stage.Stage;
import javafx.fxml.FXMLLoader;
\end{lstlisting}

\textbf{Všechny použité knihovny jsou součástí Java Standard Edition.} Nebyly použity žádné externí networking knihovny.

\subsection{Timeouty a reconnect}

\begin{table}[H]
\centering
\begin{tabular}{|l|r|p{5cm}|}
\hline
\textbf{Konstanta} & \textbf{Hodnota} & \textbf{Popis} \\
\hline
CONNECTION\_TIMEOUT & 5s & Timeout při navazování spojení \\
READ\_TIMEOUT & 15s & Timeout při čtení ze socketu \\
RECONNECT\_INTERVAL & 10s & Interval mezi reconnect pokusy \\
MAX\_RECONNECT\_ATTEMPTS & 7 & Počet pokusů (= 70s celkem) \\
\hline
\end{tabular}
\caption{Timeouty klienta}
\end{table}

\subsection{GUI a uživatelské rozhraní}

Klient implementuje \textbf{grafické rozhraní pomocí JavaFX} (ne konzole):

\begin{description}
    \item[Login obrazovka] -- Zadání IP, portu a přezdívky
    \item[Lobby obrazovka] -- Seznam místností, možnost vytvořit/připojit se
    \item[Game obrazovka] -- Herní deska, karty, skóre, informace o hráčích
\end{description}

\textbf{Klíčové vlastnosti GUI:}
\begin{itemize}[noitemsep]
    \item Non-blocking -- síťové operace v odděleném threadu
    \item Aktuální stav hry -- herní pole, přezdívky, kdo je na tahu
    \item Indikace nedostupnosti -- server/protihráč offline
    \item Validace vstupů -- IP adresa, port, nickname, herní tahy
\end{itemize}

\subsection{Logování}

Klient loguje události do souboru \texttt{client.log}:

\begin{itemize}[noitemsep]
    \item Připojení k serveru
    \item Autentizace (client\_id)
    \item Reconnect události
    \item Chyby (síťové, protokolové)
    \item Změny stavů
\end{itemize}

\textbf{Příklad logu:}
\begin{lstlisting}
[2026-01-11 12:00:00] [INFO] Logger initialized
[2026-01-11 12:00:00] [INFO] Application starting...
[2026-01-11 12:00:05] [INFO] Connecting to 127.0.0.1:10000
[2026-01-11 12:00:05] [INFO] Connected successfully
[2026-01-11 12:00:06] [INFO] Client authenticated (ID: 1)
[2026-01-11 12:01:30] [WARNING] Connection timeout - attempting reconnect
[2026-01-11 12:01:40] [INFO] Reconnect successful after 1 attempts
\end{lstlisting}

\newpage

% 5. PŘEKLAD A SPUŠTĚNÍ
\section{Překlad a spuštění}

\subsection{Požadavky na prostředí}

\textbf{Server (C):}
\begin{itemize}[noitemsep]
    \item GNU/Linux
    \item GCC 7.5+ nebo Clang 6.0+
    \item GNU Make 4.1+
    \item POSIX threads (libpthread)
\end{itemize}

\textbf{Klient (Java):}
\begin{itemize}[noitemsep]
    \item Java SE 17+ (OpenJDK nebo Oracle JDK)
    \item Apache Maven 3.6+
    \item JavaFX 17+ (stáhne Maven automaticky)
\end{itemize}

\subsection{Překlad serveru}

\textbf{Pomocí Makefile:}
\begin{lstlisting}[language=bash]
cd server_src
make clean    # Vycistit stare buildy
make          # Zkompilovat server
\end{lstlisting}

\textbf{Makefile:}
\begin{lstlisting}[language=make]
CC = gcc
CFLAGS = -Wall -Wextra -pthread -g
SOURCES = main.c server.c client_handler.c client_list.c \
          logger.c room.c game.c
TARGET = server

all:
    $(CC) $(CFLAGS) $(SOURCES) -o $(TARGET)

clean:
    rm -f $(TARGET)
\end{lstlisting}

\subsection{Překlad klienta}

\textbf{Pomocí Maven:}
\begin{lstlisting}[language=bash]
cd client_src
mvn clean package
\end{lstlisting}

Maven vytvoří spustitelný JAR:
\begin{lstlisting}
target/pexeso-client-1.0-SNAPSHOT.jar
\end{lstlisting}

\subsection{Spuštění}

\textbf{Server:}
\begin{lstlisting}[language=bash]
./server <IP> <PORT> <MAX_ROOMS> <MAX_CLIENTS>

# Priklad:
./server 0.0.0.0 10000 10 50
\end{lstlisting}

\textbf{Klient:}
\begin{lstlisting}[language=bash]
java -jar target/pexeso-client-1.0-SNAPSHOT.jar

# Nebo pres Maven:
mvn javafx:run
\end{lstlisting}

\newpage

% 6. TESTOVÁNÍ
\section{Testování}

\subsection{Testovací scénáře}

\textbf{1. Základní hra (2 hráči):}
\begin{enumerate}[noitemsep]
    \item Alice se připojí, vytvoří místnost
    \item Bob se připojí, vstoupí do místnosti
    \item Alice spustí hru
    \item Hráči střídavě otáčejí karty
    \item Hra skončí, zobrazí se výsledky
\end{enumerate}

\textbf{2. Fragmentace zpráv (InTCPtor):}
\begin{itemize}[noitemsep]
    \item Zprávy přicházejí po 1--2 bajtech
    \item Server/klient správně skládá fragmenty
    \item Hra běží bez chyb (jen pomaleji)
\end{itemize}

\textbf{3. Reconnect mechanismus:}
\begin{itemize}[noitemsep]
    \item Simulace krátkodobého výpadku (< 90s)
    \item Automatický reconnect klienta
    \item Obnovení stavu hry (GAME\_STATE)
    \item Simulace dlouhodobého výpadku (> 90s)
    \item Odebrání hráče ze hry
\end{itemize}

\textbf{4. Nevalidní zprávy:}
\begin{itemize}[noitemsep]
    \item Náhodná data: \texttt{cat /dev/urandom | nc 127.0.0.1 10000}
    \item Malformed příkazy: \texttt{HELLO}, \texttt{FLIP} (bez parametrů)
    \item Příkazy ve špatném stavu: \texttt{FLIP} před \texttt{HELLO}
    \item Server loguje chyby a odpojí po 3 chybách
\end{itemize}

\textbf{5. Memory leaks (Valgrind):}
\begin{lstlisting}[language=bash]
valgrind --leak-check=full ./server 0.0.0.0 10000 10 50
\end{lstlisting}

Očekávaný výsledek: \texttt{0 bytes definitely lost}

\subsection{Výsledky testování}

\begin{table}[H]
\centering
\begin{tabular}{|l|c|}
\hline
\textbf{Test} & \textbf{Výsledek} \\
\hline
Kompletní hra (2--4 hráči) & $\checkmark$ \\
Fragmentace zpráv (extrémní) & $\checkmark$ \\
Reconnect SHORT (< 90s) & $\checkmark$ \\
Reconnect LONG (> 90s) & $\checkmark$ \\
Nevalidní zprávy & $\checkmark$ \\
Memory leaks (valgrind) & $\checkmark$ (0 leaks) \\
Paralelní místnosti & $\checkmark$ \\
PING/PONG keepalive & $\checkmark$ \\
GUI non-blocking & $\checkmark$ \\
\hline
\end{tabular}
\caption{Výsledky testování}
\end{table}

\newpage

% 7. ZÁVĚR
\section{Závěr}

\subsection{Dosažené výsledky}

Projekt splňuje všechny požadavky zadání:

\begin{itemize}[noitemsep]
    \item \textbf{Kompletní síťová hra} -- 2--4 hráči, lobby, místnosti
    \item \textbf{Textový protokol} -- ASCII přes TCP, čitelný a rozšiřitelný
    \item \textbf{Robustní implementace} -- validace zpráv, error handling, logování
    \item \textbf{Reconnect mechanismus} -- automatický při krátkodobém výpadku
    \item \textbf{Paralelní běh} -- více místností současně bez vzájemného ovlivňování
    \item \textbf{Stabilita} -- bez memory leaks (ověřeno valgrind), bez segfaultů
    \item \textbf{Modulární kód} -- čitelná struktura, dokumentované komentáři
    \item \textbf{Standardní build} -- Makefile (server), Maven (klient)
\end{itemize}

\subsection{Použité technologie}

\textbf{Server:}
\begin{itemize}[noitemsep]
    \item Jazyk: C
    \item Síťování: BSD sockets (POSIX)
    \item Paralelizace: POSIX threads (thread-per-client)
    \item Build: GNU Make
\end{itemize}

\textbf{Klient:}
\begin{itemize}[noitemsep]
    \item Jazyk: Java 17
    \item Síťování: java.net.Socket (Java SE)
    \item GUI: JavaFX 17
    \item Build: Apache Maven
\end{itemize}

\subsection{Možná rozšíření}

V budoucnosti lze projekt rozšířit o:

\begin{itemize}[noitemsep]
    \item Chat mezi hráči v místnosti
    \item Statistiky hráčů (počet her, výher, průměrné skóre)
    \item Žebříček nejlepších hráčů
    \item Spectator režim (sledování hry bez účasti)
    \item Registrace a autentizace hráčů (přezdívka + heslo)
    \item Různé velikosti herních desek (konfigurovatelné při vytvoření místnosti)
\end{itemize}

\subsection{Zhodnocení}

Implementace prokázala schopnost navrhnout a realizovat kompletní síťovou aplikaci s robustním protokolem, automatickým zotavením po výpadcích a stabilním během. Projekt splňuje všechny požadavky zadání a byl úspěšně otestován na různých scénářích včetně extrémní fragmentace zpráv a síťových výpadků.

Klíčové poznatky:
\begin{itemize}[noitemsep]
    \item Důležitost správného buffering při fragmentaci zpráv
    \item Nutnost thread-safe operací při paralelním běhu
    \item Význam keepalive mechanismu pro detekci výpadků
    \item Potřeba důkladné validace všech vstupů (bezpečnost)
\end{itemize}

\end{document}
