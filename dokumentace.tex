\documentclass[12pt,a4paper]{article}
\usepackage[utf8]{inputenc}
\usepackage[czech]{babel}
\usepackage[T1]{fontenc}
\usepackage{geometry}
\geometry{margin=2.5cm}
\usepackage{graphicx}
\usepackage{listings}
\usepackage{xcolor}
\usepackage{hyperref}
\usepackage{enumitem}
\usepackage{float}
\usepackage{fancyhdr}
\usepackage{amsmath}
\usepackage{amssymb}

\lstset{
    basicstyle=\ttfamily\small,
    breaklines=true,
    frame=single,
    numbers=left,
    numberstyle=\tiny,
    tabsize=4,
    showstringspaces=false,
    commentstyle=\color{gray},
    keywordstyle=\color{blue},
    stringstyle=\color{red}
}

\setlength{\headheight}{15pt}
\pagestyle{fancy}
\fancyhf{}
\rhead{KIV/UPS - Pexeso}
\lhead{Semestrální práce}
\rfoot{Strana \thepage}

\hypersetup{
    colorlinks=true,
    linkcolor=blue,
    filecolor=magenta,
    urlcolor=cyan,
    pdftitle={Dokumentace - Pexeso},
}

\begin{document}
\sloppy

% Titulní strana
\begin{titlepage}
    \centering
    \vspace*{2cm}
    {\Huge\bfseries Síťová hra Pexeso\par}
    \vspace{1cm}
    {\Large Semestrální práce z předmětu KIV/UPS\par}
    \vspace{2cm}
    {\Large Architektura server-klient (1:N)\par}
    \vspace{0.5cm}
    {\large Textový protokol přes TCP\par}
    \vspace{3cm}
    {\large
    Server: C (BSD sockets, POSIX threads)\\
    Klient: Java (JavaFX, java.net.Socket)\\
    }
    \vspace{2cm}
    {\Large\textbf{Autor:} Oldřich Jan Švehla\par}
    {\textbf{Osobní číslo:} A23B0234P\par}
    \vfill
    {\large Západočeská univerzita v Plzni\\
    Fakulta aplikovaných věd\\
    \today\par}
\end{titlepage}

\thispagestyle{empty}
\tableofcontents
\newpage
\setcounter{page}{1}

%===============================================================================
% 1. ÚVOD A POPIS HRY
%===============================================================================
\section{Úvod}

Tato dokumentace popisuje implementaci síťové hry Pexeso (Memory) jako semestrální práce z předmětu KIV/UPS. Projekt implementuje kompletní client-server architekturu s textovým protokolem nad TCP.

\subsection{Popis hry}

Pexeso je tahová paměťová hra pro 2--4 hráče. Herní deska obsahuje sudý počet karet (16, 36 nebo 64) rozmístěných lícem dolů v mřížce. Každá karta má přidělenou hodnotu (číslo/symbol) a každá hodnota se na desce vyskytuje přesně dvakrát.

\textbf{Pravidla hry:}
\begin{enumerate}[noitemsep]
    \item Hráči se střídají v tazích podle pořadí určeného serverem
    \item V každém tahu hráč otočí postupně 2 karty (příkazy \texttt{FLIP})
    \item Pokud se hodnoty obou karet shodují:
    \begin{itemize}[noitemsep]
        \item Hráč získává 1 bod
        \item Karty zůstávají otočené (jsou odebrány ze hry)
        \item Hráč pokračuje dalším tahem
    \end{itemize}
    \item Pokud se hodnoty neshodují:
    \begin{itemize}[noitemsep]
        \item Karty se po krátké prodlevě vrátí lícem dolů
        \item Na tahu je další hráč v pořadí
    \end{itemize}
    \item Hra končí, když jsou nalezeny všechny páry
    \item Vyhrává hráč s nejvyšším skóre (při remíze více vítězů)
\end{enumerate}

\subsection{Architektura systému}

Systém používá architekturu \textbf{server-klient (1:N)} s centrálním serverem obsluhujícím více klientů současně.

\begin{description}[leftmargin=2.5cm]
    \item[Server (C)] Autoritativní server spravující veškerou herní logiku, místnosti, připojené klienty a stavy her. Implementován v jazyce C s využitím BSD sockets \newline a POSIX threads.
    \item[Klient (Java)] Grafická aplikace s JavaFX pro zobrazení hry a interakci s uživatelem. Komunikuje se serverem pomocí textového protokolu přes TCP.
\end{description}

\textbf{Klíčové vlastnosti systému:}
\begin{itemize}[noitemsep]
    \item Textový protokol (ASCII) přes TCP -- čitelný, snadno debugovatelný
    \item Podpora 2--4 hráčů v jedné herní místnosti
    \item Lobby systém s výběrem a vytvářením místností
    \item Automatický reconnect při krátkodobém výpadku (< 90s)
    \item PING/PONG keepalive mechanismus pro detekci výpadků
    \item Validace všech síťových zpráv (ochrana proti nevalidním datům)
    \item Logování událostí na serveru i klientovi
    \item Paralelní běh více herních místností bez vzájemného ovlivňování
\end{itemize}

\newpage

%===============================================================================
% 2. KOMUNIKAČNÍ PROTOKOL
%===============================================================================
\section{Komunikační protokol}

Tato sekce popisuje protokol dostatečně pro implementaci alternativního klienta nebo serveru.

\subsection{Formát zpráv}

Protokol je navržen jako \textbf{textový aplikační protokol} nad transportním protokolem TCP.

\begin{table}[H]
\centering
\begin{tabular}{|l|l|}
\hline
\textbf{Parametr} & \textbf{Hodnota} \\
\hline
Typ protokolu & Textový, aplikační vrstva \\
Transportní protokol & TCP \\
Kódování & ASCII (bez diakritiky, znaky 0x20--0x7E) \\
Ukončení zprávy & \texttt{\textbackslash n} (newline, 0x0A) \\
Formát zprávy & \texttt{COMMAND [PARAM1] [PARAM2] ...} \\
Oddělování parametrů & Jedna mezera (ASCII 0x20) \\
Maximální délka & 8192 bajtů \\
\hline
\end{tabular}
\caption{Základní parametry protokolu}
\end{table}

\textbf{Příklad zprávy:} \texttt{FLIP 5\textbackslash n} -- hráč otáčí kartu s indexem 5.

\subsection{Přenášené datové typy}

\begin{table}[H]
\centering
\small
\begin{tabular}{|l|p{5cm}|p{5cm}|}
\hline
\textbf{Typ} & \textbf{Formát} & \textbf{Omezení} \\
\hline
nickname & String bez mezer & 1--16 znaků, povolené: a-z, \newline A-Z, 0-9, \_ \\
room\_name & String bez mezer & 1--20 znaků \\
room\_id & Celé číslo (Integer) & 1--9999 \\
client\_id & Celé číslo (Integer) & 1--9999 \\
card\_id & Celé číslo (Integer) & 0 až (board\_size - 1) \\
card\_value & Celé číslo (Integer) & 0 až (board\_size / 2 - 1) \\
board\_size & Celé číslo (Integer) & 16, 36, nebo 64 (4×4, 6×6, 8×8) \\
max\_players & Celé číslo (Integer) & 2--4 \\
score & Celé číslo (Integer) & $\geq$ 0 \\
state & String & WAITING, PLAYING, \newline FINISHED \\
\hline
\end{tabular}
\caption{Přenášené datové typy a jejich omezení}
\end{table}

\subsection{Příkazy klient \texorpdfstring{$\rightarrow$}{->} server (10 příkazů)}

\begin{table}[H]
\centering
\small
\begin{tabular}{|l|l|p{5.5cm}|}
\hline
\textbf{Příkaz} & \textbf{Formát} & \textbf{Význam} \\
\hline
HELLO & \texttt{HELLO <nick>} & Autentizace hráče s přezdívkou \\
LIST\_ROOMS & \texttt{LIST\_ROOMS} & Žádost o seznam místností \\
CREATE\_ROOM & \texttt{CREATE\_ROOM <name> <max> <size>} & Vytvoření nové místnosti \\
JOIN\_ROOM & \texttt{JOIN\_ROOM <room\_id>} & Vstup do existující místnosti \\
LEAVE\_ROOM & \texttt{LEAVE\_ROOM} & Odchod z místnosti \\
START\_GAME & \texttt{START\_GAME} & Spuštění hry (pouze owner) \\
READY & \texttt{READY} & Potvrzení připravenosti ke hře \\
FLIP & \texttt{FLIP <card\_id>} & Otočení karty na dané pozici \\
PONG & \texttt{PONG} & Odpověď na PING (keepalive) \\
RECONNECT & \texttt{RECONNECT <client\_id>} & Znovupřipojení po výpadku \\
\hline
\end{tabular}
\caption{Příkazy od klienta k serveru}
\end{table}

\subsection{Příkazy server \texorpdfstring{$\rightarrow$}{->} klient (23 příkazů)}

\begin{table}[H]
\centering
\small
\begin{tabular}{|l|p{8.5cm}|}
\hline
\textbf{Příkaz} & \textbf{Formát a význam} \\
\hline
WELCOME & \texttt{WELCOME <client\_id>} -- Úspěšná autentizace, \newline přidělené ID \\
ROOM\_LIST & \texttt{ROOM\_LIST <count> [<id> <name> <curr> <max> <state> <size>]*} \\
ROOM\_CREATED & \texttt{ROOM\_CREATED <room\_id> <name>} -- Místnost  \newline vytvořena \\
ROOM\_JOINED & \texttt{ROOM\_JOINED <room\_id> <name>} -- Úspěšný vstup \\
PLAYER\_JOINED & \texttt{PLAYER\_JOINED <nick>} -- Jiný hráč vstoupil \\
PLAYER\_LEFT & \texttt{PLAYER\_LEFT <nick>} -- Jiný hráč odešel \\
PLAYER\_READY & \texttt{PLAYER\_READY <nick>} -- Hráč je připraven \\
PLAYER\_RECONNECTED & \texttt{PLAYER\_RECONNECTED <nick>} -- Hráč se znovu  \newline připojil \\
PLAYER\_DISCONNECTED & \texttt{PLAYER\_DISCONNECTED <nick> SHORT|LONG} --  \newline Výpadek \\
ROOM\_OWNER\_CHANGED & \texttt{ROOM\_OWNER\_CHANGED <nick>} -- Nový vlastník místnosti \\
GAME\_CREATED & \texttt{GAME\_CREATED <board\_size>} -- Hra připravena \\
GAME\_START & \texttt{GAME\_START <size> <player1> <player2> ...} -- Hra začíná \\
GAME\_STATE & \texttt{GAME\_STATE <size> <curr> <p1> <s1> ... <cards>} -- Stav hry \\
GAME\_END & \texttt{GAME\_END <p1> <s1> <p2> <s2> ...} -- Konec hry \\
GAME\_END\_FORFEIT & \texttt{GAME\_END\_FORFEIT <p1> <s1> ...} -- Konec  \newline vzdáním \\
YOUR\_TURN & \texttt{YOUR\_TURN} -- Jsi na tahu \\
CARD\_REVEAL & \texttt{CARD\_REVEAL <index> <value> <player>} -- \newline Otočená karta \\
MATCH & \texttt{MATCH <player> <score>} -- Shoda karet, nové skóre \\
MISMATCH & \texttt{MISMATCH <next\_player>} -- Neshoda, další hráč \\
LEFT\_ROOM & \texttt{LEFT\_ROOM} -- Potvrzení odchodu z místnosti \\
PING & \texttt{PING} -- Keepalive dotaz \\
SERVER\_SHUTDOWN & \texttt{SERVER\_SHUTDOWN} -- Server se vypíná \\
ERROR & \texttt{ERROR <code> <message>} -- Chybová zpráva \\
\hline
\end{tabular}
\caption{Příkazy od serveru ke klientovi}
\end{table}

\newpage

\subsection{Stavový diagram klienta}

Klient prochází následujícími stavy na základě protokolových zpráv:

\begin{enumerate}[noitemsep]
    \item \texttt{DISCONNECTED} -- Nepřipojeno k serveru (počáteční stav)
    \item \texttt{CONNECTING} -- Probíhá navazování TCP spojení
    \item \texttt{CONNECTED} -- TCP spojení navázáno
    \item \texttt{AUTHENTICATING} -- Odesláno \texttt{HELLO}, čeká se na \texttt{WELCOME}
    \item \texttt{IN\_LOBBY} -- V lobby, může listovat/vytvářet místnosti
    \item \texttt{IN\_ROOM\_WAITING} -- V místnosti, čeká na další hráče
    \item \texttt{IN\_ROOM\_READY} -- Hráč potvrdil připravenost (\texttt{READY})
    \item \texttt{IN\_GAME\_WAITING} -- Hra běží, čeká na svůj tah
    \item \texttt{IN\_GAME\_MY\_TURN} -- Hra běží, hráč je na tahu
    \item \texttt{RECONNECTING} -- Pokus o automatické znovupřipojení
    \item \texttt{DISCONNECTED\_PERMANENT} -- Trvalé odpojení (koncový stav)
\end{enumerate}

\begin{figure}[H]
\centering
\includegraphics[width=\textwidth]{images/client_state_diagram.png}
\caption{Stavový diagram klienta -- návaznost zpráv}
\label{fig:state-diagram}
\end{figure}

\newpage

\subsection{Keepalive mechanismus (PING/PONG)}

Server pravidelně kontroluje živost klientů:
\begin{itemize}[noitemsep]
    \item Server posílá \texttt{PING} každých 5 sekund po přijetí \texttt{PONG}
    \item Klient musí odpovědět \texttt{PONG} do 5 sekund
    \item Pokud klient neodpoví, server ho označí jako odpojeného
    \item Klient má \texttt{READ\_TIMEOUT} 15 sekund -- pokud nepřijde žádná zpráva, detekuje výpadek
\end{itemize}

\subsection{Reconnect mechanismus}

\textbf{Krátkodobý výpadek (< 90s):}
\begin{itemize}[noitemsep]
    \item Klient automaticky zkouší reconnect: 7 pokusů × 10s = 70s celkem
    \item Server čeká 90 sekund na reconnect
    \item Ostatní hráči dostávají \texttt{PLAYER\_DISCONNECTED <nick> SHORT}
    \item Po úspěšném reconnectu (\texttt{RECONNECT <id>}) server pošle \texttt{GAME\_STATE}
    \item Hra pokračuje ze stejného stavu
\end{itemize}

\textbf{Dlouhodobý výpadek (> 90s):}
\begin{itemize}[noitemsep]
    \item Server odstraní hráče ze hry
    \item Ostatní dostávají \texttt{PLAYER\_DISCONNECTED <nick> LONG}
    \item Pokud zbývá < 2 hráči, hra končí (\texttt{GAME\_END\_FORFEIT})
    \item Klient musí provést novou autentizaci (\texttt{HELLO})
\end{itemize}

\subsection{Omezení vstupních hodnot a validace dat}

Obě aplikace validují všechny příchozí zprávy na několika úrovních.

\subsubsection{Validace na úrovni spojení}
\begin{itemize}[noitemsep]
    \item \textbf{Binární/neplatná data} -- Pouze ASCII tisknutelné znaky (0x20--0x7E)
    \item \textbf{Maximální délka zprávy} -- 8192 bajtů (ochrana proti DoS)
    \item \textbf{Známý příkaz} -- Zpráva musí začínat validním protokolovým příkazem
    \item \textbf{Počítadlo chyb} -- Po 3 nevalidních zprávách je klient/server odpojen
\end{itemize}

\subsubsection{Validace hodnot parametrů}
\begin{table}[H]
\centering
\small
\begin{tabular}{|l|l|l|}
\hline
\textbf{Parametr} & \textbf{Validace} & \textbf{Příklad nevalidní hodnoty} \\
\hline
card\_id & $0 \leq$ card\_id $<$ board\_size & \texttt{FLIP -1}, \texttt{FLIP 999} \\
card\_value & $0 \leq$ value $<$ board\_size/2 & záporná hodnota \\
score & score $\geq 0$ & záporné skóre \\
board\_size & hodnota 16, 36, nebo 64 & \texttt{GAME\_START 5 ...} \\
nickname & neprázdný, 1--16 znaků & prázdný řetězec \\
room\_id & $> 0$ & \texttt{JOIN\_ROOM 0} \\
\hline
\end{tabular}
\caption{Validace hodnot parametrů}
\end{table}

\subsection{Chybové stavy a jejich hlášení}

Server odpovídá na chyby zprávou \texttt{ERROR <code> <message>}:

\begin{table}[H]
\centering
\small
\begin{tabular}{|l|p{7cm}|}
\hline
\textbf{Kód chyby} & \textbf{Význam a kdy nastává} \\
\hline
INVALID\_COMMAND & Neznámý příkaz (překlep, neexistující \newline příkaz) \\
INVALID\_SYNTAX & Chybná syntaxe příkazu (chybějící \newline parametr) \\
INVALID\_PARAMS & Špatný počet nebo formát parametrů \\
NOT\_AUTHENTICATED & Příkaz vyžaduje předchozí \texttt{HELLO} \\
ALREADY\_AUTHENTICATED & Opakované \texttt{HELLO} \\
ROOM\_NOT\_FOUND & Místnost s daným ID neexistuje \\
ROOM\_FULL & Místnost dosáhla limitu hráčů \\
NOT\_IN\_ROOM & Hráč není v žádné místnosti \\
NOT\_ROOM\_OWNER & Pouze vlastník může spustit hru \\
INVALID\_MOVE & Neplatný herní tah (mimo tah, mimo \newline hranice) \\
INVALID\_CARD & Neplatný index karty nebo již otočená karta \\
GAME\_NOT\_STARTED & Příkaz \texttt{FLIP} mimo hru \\
\hline
\end{tabular}
\caption{Chybové kódy protokolu}
\end{table}

\textbf{Reakce na nevalidní data:}
\begin{itemize}[noitemsep]
    \item Server/klient loguje chybu (bez obsahu dat z bezpečnostních důvodů)
    \item Inkrementuje počítadlo chyb
    \item Po 3 chybách odpojí protistranu (klient bez auto-reconnect)
    \item Server odpovídá \texttt{ERROR}, klient ignoruje nevalidní zprávy
\end{itemize}

\newpage

%===============================================================================
% 3. IMPLEMENTACE SERVERU
%===============================================================================
\section{Implementace serveru (C)}

\subsection{Dekompozice do modulů}

Server je implementován v jazyce C s následující modulární strukturou:

\begin{table}[H]
\centering
\small
\begin{tabular}{|l|p{8cm}|}
\hline
\textbf{Modul (soubor)} & \textbf{Odpovědnost} \\
\hline
\texttt{main.c} & Vstupní bod aplikace, parsování argumentů \newline příkazové řádky, inicializace a spuštění serveru \\
\texttt{server.c/.h} & Správa listening socketu, accept loop \newline pro příchozí spojení, vytváření klientských threadů \\
\texttt{client\_handler.c/.h} & Obsluha jednotlivých klientů, zpracování \newline a parsování protokolových zpráv, state machine klienta \\
\texttt{client\_list.c/.h} & Správa seznamu všech připojených klientů,\newline vyhledávání podle ID/nickname \\
\texttt{room.c/.h} & Správa herních místností, lobby funkcionalita, broadcast zpráv hráčům v místnosti \\
\texttt{game.c/.h} & Herní logika Pexeso -- inicializace desky,\newline zpracování tahů, vyhodnocení shody, skóre \\
\texttt{logger.c/.h} & Thread-safe logování do souboru \texttt{server.log} \\
\texttt{protocol.h} & Definice protokolových konstant a příkazů \\
\hline
\end{tabular}
\caption{Moduly serveru a jejich odpovědnosti}
\end{table}

\subsection{Rozvrstvení aplikace}

\begin{verbatim}
+------------------+
|    main.c        |  <- Vstupní bod, konfigurace
+------------------+
         |
+------------------+
|    server.c      |  <- Síťová vrstva (accept, socket)
+------------------+
         |
+------------------+
| client_handler.c |  <- Protokolová vrstva (parsování zpráv)
+------------------+
         |
    +----+----+
    |         |
+-------+ +-------+
| room.c| | game.c|  <- Aplikační vrstva (herní logika)
+-------+ +-------+
\end{verbatim}

\newpage

\subsection{Metoda paralelizace -- POSIX threads}

Server používá model \textbf{thread-per-client}:

\begin{itemize}[noitemsep]
    \item \textbf{Hlavní thread} -- běží v \texttt{server\_run()}, nekonečná accept loop čekající na nová spojení
    \item \textbf{Klientské thready} -- pro každého nového klienta se vytvoří vlastní thread pomocí \texttt{pthread\_create()}
    \item Thread se detachuje (\texttt{pthread\_detach}) pro automatický cleanup po ukončení
    \item Každý thread má vlastní \texttt{client\_t} strukturu s kontextem klienta
\end{itemize}

\textbf{Synchronizace:}
\begin{itemize}[noitemsep]
    \item Logger používá \texttt{pthread\_mutex\_t} pro thread-safe zápis do souboru
    \item Room broadcast operace jsou chráněny mutexem místnosti
    \item Seznam klientů je chráněn globálním mutexem
\end{itemize}

\subsection{Použité knihovny}

\begin{itemize}[noitemsep]
    \item \textbf{BSD Sockets (POSIX):} \texttt{sys/socket.h}, \texttt{netinet/in.h}, \texttt{arpa/inet.h} -- \texttt{socket()}, \texttt{bind()}, \texttt{listen()}, \texttt{accept()}, \texttt{send()}, \texttt{recv()}
    \item \textbf{POSIX Threads:} \texttt{pthread.h} -- \texttt{pthread\_create()}, \texttt{pthread\_detach()}, \texttt{pthread\_mutex\_t}
    \item \textbf{Standardní knihovna C:} \texttt{unistd.h}, \texttt{string.h}, \texttt{stdlib.h}, \texttt{stdio.h}, \texttt{time.h}
\end{itemize}

\textbf{Žádné externí knihovny pro síťování nebo serializaci nebyly použity.}

\subsection{Konfigurace serveru}

Server přijímá konfiguraci z příkazové řádky:
\begin{lstlisting}[language=bash]
./server <IP> <PORT> <MAX_ROOMS> <MAX_CLIENTS>
\end{lstlisting}

\begin{itemize}[noitemsep]
    \item \texttt{IP} -- IP adresa pro bind (\texttt{0.0.0.0} pro všechna rozhraní)
    \item \texttt{PORT} -- Port pro naslouchání (např. \texttt{10000})
    \item \texttt{MAX\_ROOMS} -- Maximální počet herních místností
    \item \texttt{MAX\_CLIENTS} -- Celkový limit připojených klientů
\end{itemize}

\newpage

%===============================================================================
% 4. IMPLEMENTACE KLIENTA
%===============================================================================
\section{Implementace klienta (Java + JavaFX)}

\subsection{Dekompozice do tříd}

Klient je implementován v Javě s architekturou Model-View-Controller:
\begin{table}[H]
\centering
\small
\begin{tabular}{|l|p{8cm}|}
\hline
\textbf{Třída/Balíček} & \textbf{Odpovědnost} \\
\hline
\texttt{Main} & Vstupní bod aplikace, inicializace JavaFX, správa scén \\
\texttt{network/ClientConnection} & TCP spojení, asynchronní čtení v odděleném threadu, auto-reconnect logika \\
\texttt{network/MessageListener} & Rozhraní (interface) pro příjem zpráv a událostí \\
\texttt{protocol/ProtocolConstants} & Definice protokolových příkazů a timeoutů \\
\texttt{model/Room} & Datový model herní místnosti \\
\texttt{controller/LoginController} & Ovládání přihlašovací obrazovky \\
\texttt{controller/LobbyController} & Ovládání lobby (seznam místností) \\
\texttt{controller/GameController} & Ovládání herní obrazovky, zpracování herních zpráv \\
\texttt{util/Logger} & Thread-safe logování do \texttt{client.log} \\
\hline
\end{tabular}
\caption{Třídy klienta a jejich odpovědnosti}
\end{table}

\subsection{Rozvrstvení aplikace (MVC)}

\begin{verbatim}
+---------------------------+
|     View (FXML + CSS)     |  <- UI definice
+---------------------------+
            |
+---------------------------+
|   Controller (JavaFX)     |  <- Logika UI, reakce na události
+---------------------------+
            |
+---------------------------+
|   Model + Network         |  <- Datové struktury, síťová komunikace
+---------------------------+
\end{verbatim}

\newpage

\subsection{Síťová komunikace}

Klient používá \textbf{blokující I/O} s odděleným vláknem pro čtení (reader thread):

\begin{lstlisting}[language=Java]
// Reader thread - bezi nezavisle na GUI
private void readerLoop() {
    while (running && (line = in.readLine()) != null) {
        final String message = line.trim();
        if (isValidMessage(message) && isValidProtocolCommand(message)) {
            if (listener != null) {
                listener.onMessageReceived(message);
            }
        }
    }
}
\end{lstlisting}

GUI aktualizace probíhají přes \texttt{Platform.runLater()} pro thread-safety.

\subsection{Použité knihovny}

\begin{itemize}[noitemsep]
    \item \textbf{Java Standard Library:} \texttt{java.net.Socket}, \texttt{java.net.InetSocketAddress}, \texttt{java.io.BufferedReader}, \texttt{java.io.PrintWriter}
    \item \textbf{JavaFX 17:} \texttt{javafx.application.*}, \texttt{javafx.scene.*}, \texttt{javafx.fxml.*}, \texttt{javafx.stage.*}
\end{itemize}

\textbf{Žádné externí knihovny pro síťování nebo serializaci nebyly použity.}

\subsection{Timeouty}

\begin{table}[H]
\centering
\begin{tabular}{|l|r|p{5cm}|}
\hline
\textbf{Konstanta} & \textbf{Hodnota} & \textbf{Popis} \\
\hline
CONNECTION\_TIMEOUT & 5 s & Timeout při navazování TCP spojení \\
READ\_TIMEOUT & 15 s & Timeout při čtení ze socketu \\
RECONNECT\_INTERVAL & 10 s & Interval mezi reconnect \newline pokusy \\
MAX\_RECONNECT\_ATTEMPTS & 7 & Počet pokusů (celkem 70 s) \\
\hline
\end{tabular}
\caption{Timeouty klienta}
\end{table}

\newpage

\subsection{GUI -- grafické uživatelské rozhraní}

Klient implementuje grafické rozhraní pomocí JavaFX (ne konzole):

\begin{description}[noitemsep]
    \item[Login obrazovka] -- Zadání IP adresy/hostname, portu a přezdívky
    \item[Lobby obrazovka] -- Seznam místností, vytvoření nové, připojení k existující
    \item[Game obrazovka] -- Herní deska s kartami, skóre hráčů, indikace tahu
\end{description}

\textbf{Klíčové vlastnosti GUI:}
\begin{itemize}[noitemsep]
    \item \textbf{Non-blocking} -- síťové operace v odděleném threadu, GUI nezamrzá
    \item \textbf{Aktuální stav} -- zobrazuje hrací pole, přezdívky, kdo je na tahu
    \item \textbf{Indikace výpadků} -- informuje o nedostupnosti serveru/protihráče
    \item \textbf{Validace vstupů} -- kontrola IP, portu, přezdívky před odesláním
\end{itemize}

\newpage

%===============================================================================
% 5. PŘEKLAD A SPUŠTĚNÍ
%===============================================================================
\section{Požadavky na překlad a spuštění}

\subsection{Požadavky na prostředí}

\textbf{Server (C):}
\begin{itemize}[noitemsep]
    \item Operační systém: GNU/Linux
    \item Kompilátor: GCC 7.5+ nebo Clang 6.0+
    \item Build systém: GNU Make 4.1+
    \item Knihovny: POSIX threads (libpthread) -- součást systému
\end{itemize}

\textbf{Klient (Java):}
\begin{itemize}[noitemsep]
    \item Java SE 17+ (OpenJDK nebo Oracle JDK)
    \item Apache Maven 3.6+
    \item JavaFX 17+ (staženo automaticky přes Maven)
    \item Operační systém: GNU/Linux, MS Windows, macOS
\end{itemize}

\subsection{Postup překladu serveru}

\begin{lstlisting}[language=bash]
cd server_src
make clean    # Vycistit predchozi build
make          # Zkompilovat server
\end{lstlisting}

\subsection{Postup překladu klienta}

\begin{lstlisting}[language=bash]
cd client_src
mvn clean package    # Prelozit a zabalit do JAR
\end{lstlisting}

Maven vytvoří spustitelný JAR: \texttt{target/pexeso-client-1.0-SNAPSHOT.jar}

\subsection{Spuštění aplikací}

\textbf{Server:}
\begin{lstlisting}[language=bash]
./server 0.0.0.0 10000 10 50
# IP=0.0.0.0, PORT=10000, MAX_ROOMS=10, MAX_CLIENTS=50
\end{lstlisting}

\textbf{Klient:}
\begin{lstlisting}[language=bash]
java -jar target/pexeso-client-1.0-SNAPSHOT.jar
# nebo pres Maven:
mvn javafx:run
# nebo:
mvn exec:java
\end{lstlisting}

%===============================================================================
% 6. ZÁVĚR
%===============================================================================
\section{Závěr}

\subsection{Dosažené výsledky}

Projekt splňuje všechny požadavky zadání:

\begin{itemize}[noitemsep]
    \item \textbf{Kompletní síťová hra} -- 2--4 hráči, lobby systém, herní místnosti
    \item \textbf{Textový protokol} -- ASCII přes TCP, čitelný a snadno rozšiřitelný
    \item \textbf{Robustní implementace} -- validace zpráv, error handling, logování
    \item \textbf{Reconnect mechanismus} -- automatický při krátkodobém výpadku
    \item \textbf{Paralelní běh} -- více místností současně bez vzájemného ovlivňování
    \item \textbf{Stabilita} -- bez memory leaks (ověřeno valgrind), bez segfaultů
    \item \textbf{Modulární kód} -- čitelná struktura, komentáře
    \item \textbf{Standardní build} -- Makefile (server), Maven (klient)
\end{itemize}

\subsection{Použité technologie (souhrn)}

\begin{table}[H]
\centering
\begin{tabular}{|l|l|l|}
\hline
& \textbf{Server} & \textbf{Klient} \\
\hline
Jazyk & C & Java 17 \\
Síťování & BSD sockets (POSIX) & java.net.Socket \\
Paralelizace & POSIX threads & Java threads \\
GUI & -- & JavaFX 17 \\
Build & GNU Make & Apache Maven \\
\hline
\end{tabular}
\caption{Přehled použitých technologií}
\end{table}

\subsection{Zhodnocení}

Implementace prokázala schopnost navrhnout a realizovat kompletní síťovou aplikaci s robustním protokolem, automatickým zotavením po výpadcích a stabilním během. Aplikace byla otestována na scénářích zahrnujících: kompletní hru 2--4 hráčů, extrémní fragmentaci TCP zpráv, krátkodobé i dlouhodobé výpadky spojení, nevalidní/binární data a memory leaks (valgrind).

\textbf{Klíčové poznatky:}
\begin{itemize}[noitemsep]
    \item Důležitost správného bufferingu při fragmentaci TCP zpráv
    \item Nutnost thread-safe operací při paralelním běhu (mutexy)
    \item Význam keepalive mechanismu pro včasnou detekci výpadků
    \item Potřeba důkladné validace všech vstupů (bezpečnost)
\end{itemize}

\end{document}
